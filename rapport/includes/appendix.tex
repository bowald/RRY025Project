\section{Appendix: Matlab code}
%\begin{figure}[h!]
  \begin{lstlisting}
    % Constants
    close all;
    cut   = 6;  % cut-of frequency, steepness and width of the gaussian function.
    Yh    = 1; % maximum enhancement of high frequencies.
    Yl    = 0.5; % maximum inhibition of low frequencies.

    load('forest.mat'); %load input image

    % Zero padding constants
    n = size(forestgray,1);
    m = size(forestgray,2);
    q = 2*m - 1;
    p = 2*n - 1; 

    % log
    forest = log(forestgray);

    % Zero padding
    forest = padarray( forest , [p-n q-m],  0, 'post');   

    % Transform Fourier
    forest = fft2(forest);                                

    % Shift
    forest = fftshift(forest);                            
      
      % Setup D(x,y), the distance function
      [u, v] = meshgrid(1:q,1:p);
      centerU = ceil(q/2);
      centerV = ceil(p/2);

    for i = 1:length(Yl) %cut, yl and yh can be entered as vectors for parameter study.
      gaussianNumerator = ((u - centerU).^2 + (v - centerV).^2);
      H = 1 - exp(-(gaussianNumerator./ (2* cut(i).^2) ) ); % The filter
      H = (Yh(i) - Yl(i)) * H + Yl(i);
      
        %Filtering the image
        procForest = H.*forest;

        % Shift
        procForest = ifftshift(procForest);

        % I-Transform
        procForest = ifft2(procForest);

        % Crop
        procForest = procForest(1:n, 1:m);

        % Inverse log
        procForest = exp(procForest); %inverse log(x)

        % Result only in real values
        result = real(procForest);

        figure
        imshow(result,[])
    end
  \end{lstlisting}
 % \label{code:main}
  %\caption{Matlab code for homomorphic filtering}
%\end{figure}