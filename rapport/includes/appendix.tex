\section{Appendix}
\begin{figure}[h!]
  \begin{lstlisting}
    %%
    % Constants
    close all;
    cut   = [10 10];  % Grey/tolarance. Higher, a smaler spectra of grey values.
    c     = 1;
    Yh    = [1 1]; % > 1  %Brightness higher is darker
    Yl    = [0.01 0.8]; % < 1 %lower makes black areas more black

    load('forest.mat');

    figure
      imshow(forestgray, []) % Original image
    histi = histeq(forestgray);
    %figure
    n = size(forestgray,1);
    m = size(forestgray,2);
    q = 2*m - 1;
    p = 2*n - 1;
    % log
    forest = log(forestgray);

    % Zero padding
    forest = padarray( forest , [p-n q-m],  0, 'post');   
    % Transform
    forest = fft2(forest);                                

    % Shift
    forest = fftshift(forest);                            
      
      % Setup D(x,y)
      [u, v] = meshgrid(1:q,1:p);
      centerU = ceil(q/2);
      centerV = ceil(p/2);

    for i = 1:length(Yl)
      gaussianNumerator = ((u - centerU).^2 + (v - centerV).^2);
      H = 1 - exp(-c*(gaussianNumerator./ (2* cut(i).^2) ) );
      H = (Yh(i) - Yl(i)) * H + Yl(i);
      
        %The filtering
        procForest = H.*forest;

        % Shift
        procForest = ifftshift(procForest);

        % I-Transform
        procForest = ifft2(procForest);

        % Crop
        procForest = procForest(1:n, 1:m);

        % Inverse log
        procForest = exp(procForest); %reverse log(x + 1)

        % Result only in real values
        result = real(procForest);

        figure
        imshow(result,[])
    end
  \end{lstlisting}
  \label{code:main}
  \caption{Matlab code for homomorphic filtering}
\end{figure}