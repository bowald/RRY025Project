\section{Method}

An image $f(u,v)$ can be representet using an illumination-reflection model. $f(u,v) = i(u,v) r(u,v)$
The illumination $i(u,v)$ can be charectarize as slow changes in the frequency domain or as glooming light in the spartial domain. While the reflection $r(u,v)$ tends to be rapid changes in the frequenze domain or as edges in the spartial domain. The homomorphic filter works in the frequenzy domain and aims to filter out some of the illumination $i(u,v)$. This results in an incresment in contrast and normalized brightness. \\

\subsection{Pre-poccesing of the original image}
To apply the filter the image $i(u,v)$ needs to be transformed into frequenzy domain and since the illumination-reflection model is multiplicativ and will result in a convulotion in frequency space the image will need to be pre-proccesed before the transformation.
$\mathfrak{F}[f(u,v)] \neq \mathfrak{F}[i(u,v)] \mathfrak{F}[r(u,v)] $

Applying the logarithm on the original image results in a transformation of the illumination-reflection model from multiplicativ to additive

$ f(u,v) = i(u,v) r(u,v) \xrightarrow{\ln} \ln{f(u,v)} = \ln{i(u,v)} + \ln{r(u,v)} $

The Fourier Transform ,the image periodically and to avoid aliasing the source image is zero padded. Zero Padding the source image using matlabs $padarray$- function. To avoid edge values of the source image to affect each other.

Since the images are discrete, the Fourier Transform the image will loop the image periodically. To avoid aliasing the source image is zero padded. Zero Padding the source image using matlabs \textit{padarray}- function. The amount of zeros padded with the image is calculated using Equation 4.6-31 and 4.6-32(Gonzalez, Woods 2010, page 274) 
\begin{figure}
\begin{equation}
  P \geq 2M -1
  \label{eqn:Ppadded}
\end{equation}
\caption{4.6-31, P Resulting horisontal size of the image, M horizontal size of source image}
\end{figure}

\begin{figure}
  \begin{equation}
    Q \geq 2N -1
    \label{eqn:Qpadded}
  \end{equation}
  \caption{4.6-31, Q Resulting horisontal size of the image, N horizontal size of source image}
\end{figure}

The filter used in this project is center positioned. Since matlabs \textit{fft2}-function returns the transformed image with the origio in upper left corner, the result must be shifted to apply the filter correctly. Shifting the transformation is done using matlabs \textit{fftshift}.

\subsection{The construction of the homomorphic filter}

The filter used to enhanced the image in this paper is a highpass guassian filter with three parametes. 
    \begin{equation}
    \label{eqn:gaussian_filter}
      H(u,v) = \left( \gamma_H - \gamma_L \right) \left[ 1 - e^{- D(u,v) /2 * D_0^2}\right] + \gamma_L 
    \end{equation}

$\gamma_H$ wich sets the maximum amplitude of the filter.\\
$\gamma_L$ wich sets the lower bound of amplitude of the filter.\\
$D_0$ is sets the cut-of freqenzy, to control the stepness and width of the guassian function.

$D(u,v)$ is a function to calculate the distance from the center of the filter to each element ,$u,v$, in a matrix. The matrix has the dimensions of the filter and is twice as large as the original image
\begin{lstlisting}
  [u, v] = meshgrid(1:q,1:p);
  centerU = ceil(q/2);
  centerV = ceil(p/2);
  gaussianNumerator = ((u - centerU).^2 + (v - centerV).^2);
\end{lstlisting}


\subsection{Applying the filter and the post processing}


Multiplication